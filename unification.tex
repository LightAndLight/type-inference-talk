\documentclass[aspectratio=169]{beamer}
\usepackage{bcprules, textcomp, prftree, amsmath, tikz}
\usepackage[safe]{tipa}

\setbeamercolor{background canvas}{bg=black!3}

\newcommand{\ifte}[3]{\texttt{if} \; {#1} \; \texttt{then} \; {#2} \; \texttt{else} \; {#3}}
\newcommand{\lamann}[3]{\lambda {#1} \texttt{:} {#2} \texttt{.} \; {#3}}
\newcommand{\lam}[2]{\lambda {#1} \texttt{.} \; {#2}}
\newcommand{\true}[0]{\texttt{true}}
\newcommand{\false}[0]{\texttt{false}}
\newcommand{\Bool}[0]{\texttt{Bool}}
\newcommand{\bluebox}[1]{\tikz[baseline] \node[fill=blue!20,rectangle,anchor=text]{#1}; }
\newcommand{\greenbox}[1]{\tikz[baseline] \node[fill=green!20,rectangle,anchor=text]{#1}; }
\newcommand{\redbox}[1]{\tikz[baseline] \node[fill=red!20,rectangle,anchor=text]{#1}; }
\newcommand{\blueoutline}[1]{\tikz \node[draw=beamer@blendedblue, rectangle]{#1}; }
\newcommand{\blueboxon}[2]{ \temporal<#1>{#2}{\bluebox{$#2$}}{#2} }
\newcommand{\greenboxon}[2]{ \temporal<#1>{#2}{\greenbox{$#2$}}{#2} }
\newcommand{\redboxon}[2]{ \temporal<#1>{#2}{\redbox{$#2$}}{#2} }

\addtobeamertemplate{frametitle}{\let\insertframetitle\insertsectionhead}{}

\defbeamertemplate{section page}{minimal}[1][]{
  \begin{centering}{}
    \vskip1em\par
    \begin{beamercolorbox}[sep=12pt,center]{part title}
      \usebeamerfont{section title}\insertsection\par
    \end{beamercolorbox}
  \end{centering}
}

\setbeamertemplate{section page}[minimal]
\AtBeginSection{\frame{\sectionpage}}

\title{Type Inference \& Unification}
\author{Isaac Elliott}

\begin{document}
\beamertemplatenavigationsymbolsempty

\frame{\titlepage}

\section{STLC}

\begin{frame}
  \frametitle

  % here's the presentation of our familiar simply-typed lambda calculus

  \[
  \begin{aligned}
  Type ::= \\
   ~ & Type \rightarrow Type \\
   ~ & \Bool \\ \\
  term ::= \\
   ~ & var \\
   ~ & \lamann{var}{Type}{term} \\
   ~ & term \; term \\
   ~ & \true \\
   ~ & \false \\
   ~ & \ifte{term}{term}{term}
  \end{aligned}
  \]

\end{frame}

\begin{frame}
  \frametitle

  % typing rules

  \infrule[T-var]
    {}
    {\Gamma , \; x : T \vdash x : T}

  \infrule[T-lam-ann]
    {\Gamma , \; x : S \vdash e : T}
    {\Gamma \vdash \lamann{x}{S}{e} \;\; : \;\; S \rightarrow T}

  \infrule[T-app]
    {\Gamma \vdash f : S \rightarrow T \andalso \Gamma \vdash x : S}
    {\Gamma \vdash f \; x \;\; : \;\; T}

  \infrule[T-true]
    {}
    {\Gamma \vdash \true : \Bool}

  \infrule[T-false]
    {}
    {\Gamma \vdash \false : \Bool}

  \infrule[T-if]
    {\Gamma \vdash b : \Bool \andalso \Gamma \vdash x : T \andalso \Gamma \vdash y : T}
    {\Gamma \vdash \ifte{b}{x}{y} \;\; : \;\; T}

\end{frame}

\begin{frame}
  \frametitle

  %% what's the type of this expression?

  \[
  \lamann{\texttt{b}}{\Bool}{\ifte{\texttt{b}}{\false}{\true}}
  \]

\end{frame}

\begin{frame}
  \frametitle

  %% what's the type of this expression?

  \[
  \Bool \rightarrow \Bool
  \]

\end{frame}

\begin{frame}
  \frametitle

  %% but if I left off the annotation, would that change your answer?

  \[
  \lam{\texttt{b}}{\ifte{\texttt{b}}{\false}{\true}}
  \]

\end{frame}

\begin{frame}
  \frametitle

  % the syntax is straightforward

  \[
  \begin{aligned}
  term ::= \\
   ~ & \dots \\
   ~ & \lam{var}{term}
  \end{aligned}
  \]

\end{frame}

\begin{frame}
  \frametitle

  % let's design a typing rule for it

  \infrule[T-lam]{
    \onslide<3->{\blueboxon{3-5}{\Gamma ,\; x : \; \onslide<4->{??}} \vdash \blueboxon{3-5}{\onslide<5->{e :}}} \onslide<6->{\greenboxon{6}{T}}
  }{
    \onslide<2->{\blueboxon{2}{\Gamma} \vdash \blueboxon{2}{\lam{x}{e}} \;\; :} \;\; \onslide<7->{\greenboxon{7}{?? \rightarrow T}}
  }

\end{frame}

\begin{frame}
  \frametitle

  % Solving these kinds of problems is the essence of *type inference*
  % There is an unknown type, and we need

  \infrule[T-lam]
    {\Gamma , \; x : \; ?? \vdash e : T}
    {\Gamma \vdash \lam{x}{e} \;\; : \;\;\; ?? \rightarrow T}

\end{frame}

\section{Type Inference}

\begin{frame}
  \frametitle

  % definition
  %
  % we're going to augment the typing rules so that other parts of the expression
  % can give us hints about the type of the argument

  \textbf{infer} /\textipa{In"f@:}/ % /ɪnˈfəː/

  \textit{verb}

  \bigskip

  Deduce or conclude (something) from evidence and reasoning rather than from explicit statements.

\end{frame}

\begin{frame}
  \frametitle

  % I want to introduce this different judgement syntax, to represent inference
  % What I want to note is that the context and term are inputs

  \[
  \blueboxon{2}{\Gamma} \vdash \blueboxon{2}{x} \Rightarrow \greenboxon{3}{T}
  \]

\end{frame}

\begin{frame}
  \frametitle

  % Some of our typing rules automatically work with inference

  \infrule[T-var]
    {}
    {\Gamma , \; x : T \vdash x \Rightarrow T}

  \infrule[T-lam-ann]
    {\Gamma , \; x : S \vdash e \Rightarrow T}
    {\Gamma \vdash \lamann{x}{S}{e} \;\; \Rightarrow \;\; S \rightarrow T}

  \infrule[T-true]
    {}
    {\Gamma \vdash \true \Rightarrow \Bool}

  \infrule[T-false]
    {}
    {\Gamma \vdash \false \Rightarrow \Bool}

\end{frame}

\begin{frame}
  \frametitle

  % The cool thing with this presentation is that we can read the rules
  % very 'algorithmically'
  %
  % You read them in a clockwise manner, and they transform inputs into outputs

  \infrule[T-lam-ann]
    {\blueboxon{3}{\Gamma , \; x : S} \vdash \blueboxon{3}{e} \Rightarrow \greenboxon{4}{T}}
    {\blueboxon{2}{\Gamma} \vdash \blueboxon{2}{\lamann{x}{S}{e}} \;\; \Rightarrow \;\; \greenboxon{5}{S \rightarrow T}}

\end{frame}

\begin{frame}
  \frametitle

  % to add the rest of the rules, we're going to need some extra machinery
  %
  % start with meta variables
  %
  % these are variables that we only use during type checking. they're not part
  % of the syntax that the user interacts with (hence 'meta')

  \[
  \begin{aligned}
    meta ::= \\
    ~ & ?^0 \\
    ~ & ?^1 \\
    ~ & \dots \\
    ~ & ?^n \\ \\
    Type ::= \\
    ~ & \dots \\
    ~ & meta \; \text{(only during type-checking)}
  \end{aligned}
  \]

\end{frame}

\begin{frame}
  \frametitle

  % Here we're given an un-annotated lambda term, which means we're going to have
  % to guess it.
  %
  % We guess it by generating a new meta variable (one that hasn't been used yet)
  % to stand in for whatever the argument type will eventually be
  %
  % We infer the body as normal, and then the inferred type of the lambda is
  % a function from 'something' to the body type

  \infrule[T-lam]
    {new(?^n) \andalso \Gamma , \; x : \; ?^n \vdash e \Rightarrow T}
    {\Gamma \vdash \lam{x}{e} \;\; \Rightarrow \;\; ?^n \rightarrow T}

\end{frame}

\begin{frame}
  \frametitle

  % I would have laid all the assumptions out on the same line but they were getting
  % too big for the slide
  %
  % You can still read this in a clockwise order
  %
  % To infer the type of an application:
  %
  % 1. infer the function
  % 2. infer the argument
  %
  % Notice that we haven't looked at f's type. We know it needs to be a function type.
  % And we know the type of its argument, because we inferred a type for x. But what
  % should the return type be?

  \infrule[T-app]{
    \onslide<5->{\blueboxon{5}{\Gamma} \vdash \blueboxon{5}{x} \Rightarrow} \onslide<6->{\greenboxon{6}{T}} \andalso
    \onslide<7->{\redboxon{7}{new(?^n)}} \andalso \\
    \onslide<3->{\blueboxon{3}{\Gamma} \vdash \blueboxon{3}{f} \Rightarrow} \onslide<4->{\greenboxon{4}{S}} \andalso
    \onslide<8->{\redboxon{8}{S = T \rightarrow \; ?^n}}
  }{
    \onslide<2->{\blueboxon{2}{\Gamma} \vdash \blueboxon{2}{f \; x} \;\; \Rightarrow} \;\;
    \onslide<9->{\greenboxon{9}{?^n}}
  }

\end{frame}

\begin{frame}
  \frametitle

  \infrule[T-if]{
    \onslide<5->{\blueboxon{5}{\Gamma} \vdash \blueboxon{5}{x} \Rightarrow} \onslide<6->{\greenboxon{6}{T_1}} \andalso
    \onslide<7->{\blueboxon{7}{\Gamma} \vdash \blueboxon{7}{y} \Rightarrow} \onslide<8->{\greenboxon{8}{T_2}} \andalso
    \onslide<9->{\redboxon{9}{B = \Bool}} \\
    \onslide<3->{\blueboxon{3}{\Gamma} \vdash \blueboxon{3}{b} \Rightarrow} \onslide<4->{\greenboxon{4}{B}} \andalso
    \phantom{ \Gamma \vdash y \Rightarrow T_2 } \andalso
    \onslide<10->{\redboxon{10}{T_1 = T_2}}
  }{
    \onslide<2->{\blueboxon{2}{\Gamma} \vdash \blueboxon{2}{\ifte{b}{x}{y}} \;\; \Rightarrow} \;\;
    \onslide<11->{\greenboxon{11}{T_1}}
  }

\end{frame}

\begin{frame}
  \frametitle

  \[ \lam{b}{\ifte{b}{\false}{\true}} \]

\end{frame}

\newcommand{\highlighton}[1]{ \temporal<#1>{\color{black!50}}{\color{black}}{\color{black!50}} }
\newcommand{\changeto}[3]{ \temporal<#1>{#2}{#3}{#3} }

\newcommand{\lamslides}[0]{2-11,56-59}
\newcommand{\ifslides}[0]{12-23,34-37,42-45,50-53,55}
\newcommand{\varslides}[0]{24-33}
\newcommand{\falseslides}[0]{38-41}
\newcommand{\trueslides}[0]{46-49}
\newcommand{\constraintslides}[0]{54}

\begin{frame}[t]
  \frametitle

  \newcommand{\iftebfalsetrue}{\ifte{b}{\false}{\true}}

  \[
  \blueoutline{
  \scalebox{0.5}{
  \highlighton{\lamslides}{
  \prftree[r]
    { (\rn{T-lam}) }
    { new(?^0) }
    {
      \highlighton{\ifslides}{
      \prftree[r]
        { (\rn{T-if}) }
        { \highlighton{\varslides}{
          \prftree[r]
            { (\rn{T-var}) }
            {}
            { \highlighton{\ifslides,\varslides}{
              \Gamma , \; b : \; ?^0 \vdash b \Rightarrow \; ?^0
              }
            }
          }
        }
        { \highlighton{\falseslides}{
          \prftree[r]
            { (\rn{T-false}) }
            {}
            { \highlighton{\ifslides,\falseslides}{
              \Gamma , \; b : \; ?^0 \vdash \false \Rightarrow \Bool
              }
            }
          }
        }
        { \highlighton{\trueslides}{
          \prftree[r]
            { (\rn{T-true}) }
            {}
            { \highlighton{\ifslides,\trueslides}{
              \Gamma , \; b : \; ?^0 \vdash \true \Rightarrow \Bool
              }
            }
          }
        }
        { \highlighton{\ifslides,\constraintslides}{
          \prftree[r, noline]{}{?^0 = \Bool}
          }
        }
        { \highlighton{\ifslides,\constraintslides}{
          \prftree[r, noline]{}{\Bool = \Bool}
          }
        }
        { 
          \highlighton{\lamslides,\ifslides}{
          \Gamma , \; b : \; ?^0 \vdash \iftebfalsetrue \;\; \Rightarrow \;\; \Bool
          }
        }
      }
    }
    {
      \Gamma \vdash \lam{b}{\iftebfalsetrue} \;\; \Rightarrow \;\; ?^0 \rightarrow \Bool
    }
  }
  }
  }
  \]

  \newcommand{\xtobA}{ \blueboxon{3,4}{ \changeto{4}{x}{b} } }
  \newcommand{\etoif}{ \blueboxon{6,7}{ \changeto{7}{e}{\iftebfalsetrue} } }
  \newcommand{\ntozero}{ \changeto{10}{?^n}{?^0} }
  \newcommand{\ntozeroBlue}{ \blueboxon{9,10}{\ntozero} }
  \newcommand{\ntozeroGreen}{ \greenboxon{9,10}{\ntozero} }
  \newcommand{\ntozeroRed}{ \redboxon{9,10}{\ntozero} }
  \newcommand{\TtoBool}{ \greenboxon{57,58}{ \changeto{58}{T}{\Bool} } }

  \only<\lamslides>{
  \infrule[T-lam]{
    new(\ntozeroRed) \andalso \Gamma , \; \xtobA : \; \ntozeroBlue \vdash
    \etoif \Rightarrow \TtoBool
  }{
    \Gamma \vdash \lam{ \xtobA }{ \etoif } \;\; \Rightarrow \;\; \ntozeroGreen \rightarrow \TtoBool
  }
  }

  \newcommand{\gammatogammabA}{ \blueboxon{13,14}{ \changeto{14}{\Gamma}{\Gamma,\; b : \; ?^0} } }
  \newcommand{\btob}{ \blueboxon{16}{b} }
  \newcommand{\xtofalse}{ \blueboxon{18,19}{ \changeto{19}{x}{\false} } }
  \newcommand{\ytotrue}{ \blueboxon{21,22}{ \changeto{22}{y}{\true} } }
  \newcommand{\Btozero}{ \changeto{36}{B}{?^0} }
  \newcommand{\BtozeroGreen}{ \greenboxon{35,36}{\Btozero} }
  \newcommand{\BtozeroRed}{ \redboxon{35,36}{\Btozero} }
  \newcommand{\TOnetoBool}{ \changeto{44}{T_1}{\Bool} }
  \newcommand{\TOnetoBoolGreen}{ \greenboxon{43,44}{ \TOnetoBool } }
  \newcommand{\TOnetoBoolRed}{ \redboxon{43,44}{ \TOnetoBool } }
  \newcommand{\TTwotoBool}{ \changeto{52}{T_2}{\Bool} }
  \newcommand{\TTwotoBoolGreen}{ \greenboxon{51,52}{\TTwotoBool} }
  \newcommand{\TTwotoBoolRed}{ \redboxon{51,52}{\TTwotoBool} }

  \only<\ifslides>{
  \infrule[T-if]{
    \gammatogammabA \vdash \xtofalse \Rightarrow \TOnetoBoolGreen \andalso
    \gammatogammabA \vdash \ytotrue \Rightarrow \TTwotoBoolGreen \andalso
    \BtozeroRed = \Bool \\
    \gammatogammabA \vdash \btob \Rightarrow \BtozeroGreen \andalso
    \phantom{ \gammatogammabA \vdash \ytotrue \Rightarrow \TTwotoBoolGreen } \andalso
    \TOnetoBoolRed = \TTwotoBoolRed
  }{\gammatogammabA \vdash \ifte{\btob}{\xtofalse}{\ytotrue} \;\; \Rightarrow \;\; \TOnetoBoolGreen }
  }

  \newcommand{\xtobB}{ \blueboxon{28,29}{ \changeto{29}{x}{b} } }
  \newcommand{\ttozero}{ \changeto{32}{T}{?^0} }
  \newcommand{\ttozeroBlue}{ \blueboxon{31,32}{\ttozero} }
  \newcommand{\ttozeroGreen}{ \greenboxon{31,32}{\ttozero} }
  \newcommand{\gammatogammabB}{ \blueboxon{25,26}{ \changeto{26}{\Gamma}{\Gamma,\; b : \; \ttozeroBlue } } }

  \only<\varslides>{
    \infrule[T-var]{}{
      \gammatogammabB \vdash \xtobB \Rightarrow \ttozeroGreen
    }
  }

  \newcommand{\gammatogammabC}{ \blueboxon{39,40}{ \changeto{40}{\Gamma}{\Gamma,\; b : \; ?^0} } }

  \only<\falseslides>{
  \infrule[T-false]{}{
    \gammatogammabC \vdash \false \Rightarrow \Bool
  }
  }

  \newcommand{\gammatogammabD}{ \blueboxon{47,48}{ \changeto{48}{\Gamma}{\Gamma,\; b : \; ?^0} } }

  \only<\trueslides>{
  \infrule[T-true]{}{
    \gammatogammabD \vdash \true \Rightarrow \Bool
  }
  }

  \only<\constraintslides>{
    \[ ?^0 = \Bool \]
    \[ \Bool = \Bool \]
  }

\end{frame}

\begin{frame}
  \frametitle

  \[ ?^0 \rightarrow \Bool \]

  \bigskip

  \[ ?^0 = \Bool \]
  \[ \Bool = \Bool \]

\end{frame}

\begin{frame}
  \frametitle

  \[ \lam{b}{\ifte{b}{\false}{\true}} \]
  \[ \Bool \rightarrow \Bool \]

\end{frame}

\section{Unification}

\end{document}
