\documentclass[aspectratio=169]{beamer}
\usepackage{bcprules, textcomp, prftree, amsmath, tikz, hyperref}

\newcommand{\bgcolor}{black!3}
\setbeamercolor{background canvas}{bg=\bgcolor}

\newcommand{\ifte}[3]{\texttt{if} \; {#1} \; \texttt{then} \; {#2} \; \texttt{else} \; {#3}}
\newcommand{\lamann}[3]{\lambda {#1} \texttt{:} {#2} \texttt{.} \; {#3}}
\newcommand{\lam}[2]{\lambda {#1} \texttt{.} \; {#2}}
\newcommand{\true}[0]{\texttt{true}}
\newcommand{\false}[0]{\texttt{false}}
\newcommand{\Bool}[0]{\texttt{Bool}}
\newcommand{\whitebox}[1]{\tikz[baseline] \node[fill=\bgcolor,rectangle,anchor=text]{#1}; }
\newcommand{\bluebox}[1]{\tikz[baseline] \node[fill=blue!20,rectangle,anchor=text]{#1}; }
\newcommand{\greenbox}[1]{\tikz[baseline] \node[fill=green!20,rectangle,anchor=text]{#1}; }
\newcommand{\redbox}[1]{\tikz[baseline] \node[fill=red!20,rectangle,anchor=text]{#1}; }
\newcommand{\blueoutline}[1]{\tikz \node[draw=beamer@blendedblue, rectangle]{#1}; }
\newcommand{\boxon}[3]{ \temporal<#2>{\whitebox{$#3$}}{#1{$#3$}}{\whitebox{$#3$}} }
\newcommand{\blueboxon}[2]{ \boxon{\bluebox}{#1}{#2} }
\newcommand{\greenboxon}[2]{ \boxon{\greenbox}{#1}{#2} }
\newcommand{\redboxon}[2]{ \boxon{\redbox}{#1}{#2} }

\addtobeamertemplate{frametitle}{\let\insertframetitle\insertsectionhead}{}

\defbeamertemplate{section page}{minimal}[1][]{
  \begin{centering}{}
    \vskip1em\par
    \begin{beamercolorbox}[sep=12pt,center]{part title}
      \usebeamerfont{section title}\insertsection\par
    \end{beamercolorbox}
  \end{centering}
}

\setbeamertemplate{section page}[minimal]
\AtBeginSection{\frame{\sectionpage}}

\title{Type Inference \& Unification}
\author{Isaac Elliott}

\begin{document}
\beamertemplatenavigationsymbolsempty

\frame{\titlepage}

\section{Notation}

\begin{frame}
  \frametitle

  \[
  \begin{aligned}
  thing ::= & \\
  & stuff \\
  & more \; stuff \\
  & \dots
  \end{aligned}
  \]

\end{frame}

\begin{frame}
  \frametitle

  \infrule[rule-name]
    {assumption_1 \andalso assumption_2 \andalso \dots \andalso assumption_n}
    {conclusion}

\end{frame}

\begin{frame}
  \frametitle

  \[
  \begin{aligned}
  Context ::= & \\
  & \bullet \\
  & Context , var : Type
  \end{aligned}
  \]

\end{frame}

\begin{frame}
  \frametitle

  \[ Context \vdash term : Type \]

\end{frame}

\section{STLC}

\begin{frame}
  \frametitle

  % here's the presentation of our familiar simply-typed lambda calculus

  \[
  \begin{aligned}
  Type ::= \\
   ~ & Type \rightarrow Type \\
   ~ & \Bool \\ \\
  term ::= \\
   ~ & var \\
   ~ & \lamann{var}{Type}{term} \\
   ~ & term \; term \\
   ~ & \true \\
   ~ & \false \\
   ~ & \ifte{term}{term}{term}
  \end{aligned}
  \]

\end{frame}

\begin{frame}
  \frametitle

  % typing rules

  \infrule[T-var]
    {}
    {\Gamma , \; x : T \vdash x : T}

  \infrule[T-lam-ann]
    {\Gamma , \; x : S \vdash e : T}
    {\Gamma \vdash \lamann{x}{S}{e} \;\; : \;\; S \rightarrow T}

  \infrule[T-app]
    {\Gamma \vdash f : S \rightarrow T \andalso \Gamma \vdash x : S}
    {\Gamma \vdash f \; x \;\; : \;\; T}

  \infrule[T-true]
    {}
    {\Gamma \vdash \true : \Bool}

  \infrule[T-false]
    {}
    {\Gamma \vdash \false : \Bool}

  \infrule[T-if]
    {\Gamma \vdash b : \Bool \andalso \Gamma \vdash x : T \andalso \Gamma \vdash y : T}
    {\Gamma \vdash \ifte{b}{x}{y} \;\; : \;\; T}

\end{frame}

\begin{frame}
  \frametitle

  %% what's the type of this expression?

  \[
  \lamann{\texttt{b}}{\Bool}{\ifte{\texttt{b}}{\false}{\true}}
  \]

\end{frame}

\begin{frame}
  \frametitle

  %% what's the type of this expression?

  \[
  \Bool \rightarrow \Bool
  \]

\end{frame}

\begin{frame}
  \frametitle

  %% but if I left off the annotation, would that change your answer?

  \[
  \lam{\texttt{b}}{\ifte{\texttt{b}}{\false}{\true}}
  \]

\end{frame}

\begin{frame}
  \frametitle

  % the syntax is straightforward

  \[
  \begin{aligned}
  term ::= \\
   ~ & \dots \\
   ~ & \lam{var}{term}
  \end{aligned}
  \]

\end{frame}

\begin{frame}
  \frametitle

  % let's design a typing rule for it

  \infrule[T-lam]{
    \onslide<3->{\blueboxon{3,4}{\Gamma ,\; x : \; \onslide<4->{??}} \vdash \blueboxon{3,4}{e} :} \onslide<5->{\greenboxon{5}{T}}
  }{
    \onslide<2->{\blueboxon{2}{\Gamma} \vdash \blueboxon{2}{\lam{x}{e}} \;\; :} \;\; \onslide<6->{\greenboxon{6}{?? \rightarrow T}}
  }

\end{frame}

\section{Type Inference}

\begin{frame}
  \frametitle

  % I want to introduce this different judgement syntax, to represent inference
  % What I want to note is that the context and term are inputs

  \[
  \blueboxon{2}{\Gamma} \vdash \blueboxon{2}{x} \Rightarrow \greenboxon{3}{T}
  \]

\end{frame}

\begin{frame}
  \frametitle

  % Some of our typing rules automatically work with inference

  \infrule[T-var]
    {}
    {\Gamma , \; x : T \vdash x \Rightarrow T}

  \infrule[T-lam-ann]
    {\Gamma , \; x : S \vdash e \Rightarrow T}
    {\Gamma \vdash \lamann{x}{S}{e} \;\; \Rightarrow \;\; S \rightarrow T}

  \infrule[T-true]
    {}
    {\Gamma \vdash \true \Rightarrow \Bool}

  \infrule[T-false]
    {}
    {\Gamma \vdash \false \Rightarrow \Bool}

\end{frame}

\begin{frame}
  \frametitle

  % The cool thing with this presentation is that we can read the rules
  % very 'algorithmically'
  %
  % You read them in a clockwise manner, and they transform inputs into outputs

  \infrule[T-lam-ann]
    {\blueboxon{3}{\Gamma , \; x : S} \vdash \blueboxon{3}{e} \Rightarrow \greenboxon{4}{T}}
    {\blueboxon{2}{\Gamma} \vdash \blueboxon{2}{\lamann{x}{S}{e}} \;\; \Rightarrow \;\; \greenboxon{5}{S \rightarrow T}}

\end{frame}

\begin{frame}
  \frametitle

  % to add the rest of the rules, we're going to need some extra machinery
  %
  % start with meta variables
  %
  % these are variables that we only use during type checking. they're not part
  % of the syntax that the user interacts with (hence 'meta')

  \[
  \begin{aligned}
    meta ::= \\
    ~ & ?^0 \\
    ~ & ?^1 \\
    ~ & \dots \\
    ~ & ?^n \\ \\
    Type ::= \\
    ~ & \dots \\
    ~ & meta \; \text{(only during type-checking)}
  \end{aligned}
  \]

\end{frame}

\begin{frame}
  \frametitle

  % Here we're given an un-annotated lambda term, which means we're going to have
  % to guess it.
  %
  % We guess it by generating a new meta variable (one that hasn't been used yet)
  % to stand in for whatever the argument type will eventually be
  %
  % We infer the body as normal, and then the inferred type of the lambda is
  % a function from 'something' to the body type

  \infrule[T-lam]
    {\redboxon{3}{new(?^n)} \andalso \blueboxon{4}{\Gamma , \; x : \; ?^n} \vdash \blueboxon{4}{e} \Rightarrow \greenboxon{5}{T}}
    {\blueboxon{2}{\Gamma} \vdash \blueboxon{2}{\lam{x}{e}} \;\; \Rightarrow \;\; \greenboxon{6}{?^n \rightarrow T}}

\end{frame}

\begin{frame}
  \frametitle

  % I would have laid all the assumptions out on the same line but they were getting
  % too big for the slide
  %
  % You can still read this in a clockwise order
  %
  % To infer the type of an application:
  %
  % 1. infer the function
  % 2. infer the argument
  %
  % Notice that we haven't looked at f's type. We know it needs to be a function type.
  % And we know the type of its argument, because we inferred a type for x. But what
  % should the return type be?

  \infrule[T-app]{
    \onslide<5->{\blueboxon{5}{\Gamma} \vdash \blueboxon{5}{x} \Rightarrow} \onslide<6->{\greenboxon{6}{T}} \andalso
    \onslide<7->{\redboxon{7}{new(?^n)}} \andalso \\
    \onslide<3->{\blueboxon{3}{\Gamma} \vdash \blueboxon{3}{f} \Rightarrow} \onslide<4->{\greenboxon{4}{S}} \andalso
    \onslide<8->{\redboxon{8}{[S = T \rightarrow \; ?^n]}}
  }{
    \onslide<2->{\blueboxon{2}{\Gamma} \vdash \blueboxon{2}{f \; x} \;\; \Rightarrow} \;\;
    \onslide<9->{\greenboxon{9}{?^n}}
  }

\end{frame}

\begin{frame}
  \frametitle

  \infrule[T-if]{
    \onslide<5->{\blueboxon{5}{\Gamma} \vdash \blueboxon{5}{x} \Rightarrow} \onslide<6->{\greenboxon{6}{T_1}} \andalso
    \onslide<7->{\blueboxon{7}{\Gamma} \vdash \blueboxon{7}{y} \Rightarrow} \onslide<8->{\greenboxon{8}{T_2}} \andalso
    \onslide<9->{\redboxon{9}{[B = \Bool]}} \\
    \onslide<3->{\blueboxon{3}{\Gamma} \vdash \blueboxon{3}{b} \Rightarrow} \onslide<4->{\greenboxon{4}{B}} \andalso
    \phantom{ \Gamma \vdash y \Rightarrow T_2 } \andalso
    \onslide<10->{\redboxon{10}{[T_1 = T_2]}}
  }{
    \onslide<2->{\blueboxon{2}{\Gamma} \vdash \blueboxon{2}{\ifte{b}{x}{y}} \;\; \Rightarrow} \;\;
    \onslide<11->{\greenboxon{11}{T_1}}
  }

\end{frame}

\newcommand{\highlighton}[1]{ \temporal<#1>{\color{black!50}}{\color{black}}{\color{black!50}} }
\newcommand{\changeto}[3]{ \temporal<#1>{#2}{#3}{#3} }

\newcommand{\lamslides}[0]{2-11,56-59}
\newcommand{\ifslides}[0]{12-23,34-37,42-45,50-53,55}
\newcommand{\varslides}[0]{24-33}
\newcommand{\falseslides}[0]{38-41}
\newcommand{\trueslides}[0]{46-49}
\newcommand{\constraintslides}[0]{54}

\begin{frame}[t]
  % proof tree example
  \frametitle

  \newcommand{\iftebfalsetrue}{\ifte{b}{\false}{\true}}

  \[
  \blueoutline{
  \scalebox{0.5}{
  \highlighton{\lamslides}{
  \prftree[r]
    { (\rn{T-lam}) }
    { new(?^0) }
    {
      \highlighton{\ifslides}{
      \prftree[r]
        { (\rn{T-if}) }
        { \highlighton{\varslides}{
          \prftree[r]
            { (\rn{T-var}) }
            {}
            { \highlighton{\ifslides,\varslides}{
              \Gamma , \; b : \; ?^0 \vdash b \Rightarrow \; ?^0
              }
            }
          }
        }
        { \highlighton{\falseslides}{
          \prftree[r]
            { (\rn{T-false}) }
            {}
            { \highlighton{\ifslides,\falseslides}{
              \Gamma , \; b : \; ?^0 \vdash \false \Rightarrow \Bool
              }
            }
          }
        }
        { \highlighton{\trueslides}{
          \prftree[r]
            { (\rn{T-true}) }
            {}
            { \highlighton{\ifslides,\trueslides}{
              \Gamma , \; b : \; ?^0 \vdash \true \Rightarrow \Bool
              }
            }
          }
        }
        { \highlighton{\ifslides,\constraintslides}{
          \prftree[r, noline]{}{[?^0 = \Bool]}
          }
        }
        { \highlighton{\ifslides,\constraintslides}{
          \prftree[r, noline]{}{[\Bool = \Bool]}
          }
        }
        { 
          \highlighton{\lamslides,\ifslides}{
          \Gamma , \; b : \; ?^0 \vdash \iftebfalsetrue \;\; \Rightarrow \;\; \Bool
          }
        }
      }
    }
    {
      \Gamma \vdash \lam{b}{\iftebfalsetrue} \;\; \Rightarrow \;\; ?^0 \rightarrow \Bool
    }
  }
  }
  }
  \]

  \only<1>{\[ \lam{b}{\ifte{b}{\false}{\true}} \]}

  \newcommand{\xtobA}{ \blueboxon{3,4}{ \changeto{4}{x}{b} } }
  \newcommand{\etoif}{ \blueboxon{6,7}{ \changeto{7}{e}{\iftebfalsetrue} } }
  \newcommand{\ntozero}{ \changeto{10}{?^n}{?^0} }
  \newcommand{\ntozeroBlue}{ \blueboxon{9,10}{\ntozero} }
  \newcommand{\ntozeroGreen}{ \greenboxon{9,10}{\ntozero} }
  \newcommand{\ntozeroRed}{ \redboxon{9,10}{\ntozero} }
  \newcommand{\TtoBool}{ \greenboxon{57,58}{ \changeto{58}{T}{\Bool} } }

  \only<\lamslides>{
  \infrule[T-lam]{
    new(\ntozeroRed) \andalso \Gamma , \; \xtobA : \; \ntozeroBlue \vdash
    \etoif \Rightarrow \TtoBool
  }{
    \Gamma \vdash \lam{ \xtobA }{ \etoif } \;\; \Rightarrow \;\; \ntozeroGreen \rightarrow \TtoBool
  }
  }

  \newcommand{\gammatogammabA}{ \blueboxon{13,14}{ \changeto{14}{\Gamma}{\Gamma,\; b : \; ?^0} } }
  \newcommand{\btob}{ \blueboxon{16}{b} }
  \newcommand{\xtofalse}{ \blueboxon{18,19}{ \changeto{19}{x}{\false} } }
  \newcommand{\ytotrue}{ \blueboxon{21,22}{ \changeto{22}{y}{\true} } }
  \newcommand{\Btozero}{ \changeto{36}{B}{?^0} }
  \newcommand{\BtozeroGreen}{ \greenboxon{35,36}{\Btozero} }
  \newcommand{\BtozeroRed}{ \redboxon{35,36}{\Btozero} }
  \newcommand{\TOnetoBool}{ \changeto{44}{T_1}{\Bool} }
  \newcommand{\TOnetoBoolGreen}{ \greenboxon{43,44}{ \TOnetoBool } }
  \newcommand{\TOnetoBoolRed}{ \redboxon{43,44}{ \TOnetoBool } }
  \newcommand{\TTwotoBool}{ \changeto{52}{T_2}{\Bool} }
  \newcommand{\TTwotoBoolGreen}{ \greenboxon{51,52}{\TTwotoBool} }
  \newcommand{\TTwotoBoolRed}{ \redboxon{51,52}{\TTwotoBool} }

  \only<\ifslides>{
  \infrule[T-if]{
    \gammatogammabA \vdash \xtofalse \Rightarrow \TOnetoBoolGreen \andalso
    \gammatogammabA \vdash \ytotrue \Rightarrow \TTwotoBoolGreen \andalso
    [\BtozeroRed = \Bool] \\
    \gammatogammabA \vdash \btob \Rightarrow \BtozeroGreen \andalso
    \phantom{ \gammatogammabA \vdash \ytotrue \Rightarrow \TTwotoBoolGreen } \andalso
    [\TOnetoBoolRed = \TTwotoBoolRed]
  }{\gammatogammabA \vdash \ifte{\btob}{\xtofalse}{\ytotrue} \;\; \Rightarrow \;\; \TOnetoBoolGreen }
  }

  \newcommand{\xtobB}{ \blueboxon{28,29}{ \changeto{29}{x}{b} } }
  \newcommand{\ttozero}{ \changeto{32}{T}{?^0} }
  \newcommand{\ttozeroBlue}{ \blueboxon{25,26,31,32}{\ttozero} }
  \newcommand{\ttozeroGreen}{ \greenboxon{31,32}{\ttozero} }
  \newcommand{\gammatogammabB}{ \blueboxon{25,26}{ \changeto{26}{\Gamma}{\Gamma,\; b : \; \ttozeroBlue } } }

  \only<\varslides>{
    \infrule[T-var]{}{
      \gammatogammabB \vdash \xtobB \Rightarrow \ttozeroGreen
    }
  }

  \newcommand{\gammatogammabC}{ \blueboxon{39,40}{ \changeto{40}{\Gamma}{\Gamma,\; b : \; ?^0} } }

  \only<\falseslides>{
  \infrule[T-false]{}{
    \gammatogammabC \vdash \false \Rightarrow \Bool
  }
  }

  \newcommand{\gammatogammabD}{ \blueboxon{47,48}{ \changeto{48}{\Gamma}{\Gamma,\; b : \; ?^0} } }

  \only<\trueslides>{
  \infrule[T-true]{}{
    \gammatogammabD \vdash \true \Rightarrow \Bool
  }
  }

  \only<\constraintslides>{
    \[ ?^0 = \Bool \]
    \[ \Bool = \Bool \]
  }

\end{frame}

\begin{frame}
  \frametitle

  \[ ?^0 \rightarrow \Bool \]

  \bigskip

  \[ ?^0 = \Bool \]
  \[ \Bool = \Bool \]

\end{frame}

\begin{frame}
  \frametitle

  \[ \lam{b}{\ifte{b}{\false}{\true}} \]
  \[ \Bool \rightarrow \Bool \]

\end{frame}

\section{Unification}

\begin{frame}
  \frametitle

  \begin{itemize}
    \item Base types \\ \bigskip $\Bool, \; \texttt{Int}, \; \texttt{String}, \; \dots$ \bigskip
    \item Compound types \\ \bigskip $\rightarrow, \; \texttt{List}, \; \texttt{Either}, \; \dots$ \bigskip
    \item Metavariables \\ \bigskip $?^0, \; ?^1, \; ?^2, \; \dots$ \bigskip
  \end{itemize}

\end{frame}

\begin{frame}
  % Unification is about making things the same
  %
  % During type checking, we're generate these contraints, or equations
  %
  % Unification is the process of making them the same (into tautologies)
  % or saying that there was an error in the case that some equation is impossible
  \frametitle

  \[
  \begin{aligned}
    ?^0 & = \Bool \\
    \Bool & = \Bool \\
    \Bool & = \Bool \rightarrow \; ?^1 \\
    ?^2 \rightarrow \; ?^3 & = \; ?^2
  \end{aligned}
  \]

\end{frame}

\begin{frame}
  % We do this by way of a substitution
  %
  % a substition is a function that maps metas to types
  %
  % it maps some distinguished metas to other types, and all the other
  % metas are just mapped to themselves
  %
  % I don't care for including all the 'boring' metas, so when I write substitutions
  % I'm only going to note the interesting mappings and leave all the boring ones implied
  \frametitle

  \begin{center}Substitutions\end{center}

  \onslide<1->{
  \[
  \begin{cases}
    ?^0 \mapsto & \Bool \\
    ?^1 \mapsto & ?^2 \rightarrow \Bool \\
    ?^2 \mapsto & ?^2 \\
    \dots \\
    ?^n \mapsto & ?^n \\
  \end{cases}
  \]
  }

  \onslide<+->{
  \[
  \begin{cases}
    ?^0 \mapsto & \Bool \\
    ?^1 \mapsto & ?^2 \rightarrow \Bool \\
  \end{cases}
  \]
  }

\end{frame}

\begin{frame}
  % Here's some examples of what it looks like to apply a substitution
  %
  % You run the substitution function
  \frametitle

  \begin{center}Applying substitutions\end{center}

  \[
  s =
  \begin{cases}
    ?^0 \mapsto & \Bool \\
    ?^1 \mapsto & ?^2 \rightarrow \Bool
  \end{cases}
  \]

  \bigskip

  \[
  \begin{aligned}
  \onslide<2->{s(?^0) & \leadsto \Bool} \\
  \onslide<3->{s(?^1) & \leadsto \; ?^2 \rightarrow \Bool} \\
  \onslide<4->{s(?^2) & \leadsto \; ?^2}
  \end{aligned}
  \]

\end{frame}

\begin{frame}
  % I'll also say that you can apply a substitution to a type,
  %
  % by applying the substitiution to every meta present in the type
  \frametitle

  \begin{center}Applying substitutions\end{center}

  \[
  s =
  \begin{cases}
    ?^0 \mapsto & \Bool \\
    ?^1 \mapsto & ?^2 \rightarrow \Bool
  \end{cases}
  \]

  \bigskip

  \[
  \begin{aligned}
  \onslide<2->{s(\Bool) & \leadsto \Bool} \\
  \onslide<3->{s(?^0 \rightarrow \; ?^0) & \leadsto \; \Bool \rightarrow \Bool} \\
  \onslide<4->{s(?^1 \rightarrow \; ?^2) & \leadsto \; (?^2 \rightarrow \Bool) \rightarrow \; ?^2}
  \end{aligned}
  \]

\end{frame}

\begin{frame}
  % substitutions have an identity and composition
  \frametitle

  \[
  \begin{aligned}
    id & = \begin{cases}\end{cases} \\
    ~ \\
    (s_2 \circ s_1)(x) & = s_2(s_1(x))
  \end{aligned}
  \]

\end{frame}

\begin{frame}
  % We say a substitution 'solves' a system of equations if applying the subtitution
  % to both sides makes the equation trivially true (both sides are syntactically equal)
  \frametitle

  \begin{center}
    $s$ is a solution of $x = y$ when $s(x) = s(y)$ \\
    \bigskip
    $s$ unifies $x$ and $y$
  \end{center}

\end{frame}

\begin{frame}
  % so what's a solution to this equation?
  %
  % 0 -> Bool is one, but there are others
  %
  % there's a sense in which the first one is the 'best', it's a minimal solution
  % without any cruft
  \frametitle

  \begin{center}What's the solution?\end{center}

  \[ ?^0 \rightarrow \Bool = \Bool \rightarrow \Bool \]

  \[
  \begin{aligned}
  \onslide<2->{
  s & = 
  \begin{cases}
  ?^0 \mapsto & \Bool
  \end{cases}
  } \\
  \onslide<3->{
  s' & =
  \begin{cases}
  ?^0 \mapsto & \Bool \\
  ?^1 \mapsto & \Bool
  \end{cases}
  } \\
  \onslide<4->{
  s'' & =
  \begin{cases}
  ?^0 \mapsto & \Bool \\
  ?^1 \mapsto & \Bool \\
  ?^2 \mapsto & ?^{24} \rightarrow \; ?^{25}
  \end{cases}
  } \\
  \onslide<5->{\dots}
  \end{aligned}
  \]

\end{frame}

\begin{frame}
  % the 'best' solution is known as the 'most general' solution 
  \frametitle

  \begin{center}
    Most General (Solution/Unifier): \\
    \bigskip
    $s$ is the most general solution for an equation if all other
    solutions $s'$ can be written in terms of $s$ composed with some other
    substitution
  \end{center}

  \[
  \prftree
    {solves(s, e)}
    {\forall s'. \; \exists h. \; solves(s', e) \Rightarrow s' = s \circ h}
  \]
\end{frame}

\begin{frame}
  % so all the solutions to the earlier equation can be written in terms of `s`
  %
  % this is the first desirable property of solutions that we calculate
  \frametitle

  \[
  \begin{aligned}
  s & = 
  \begin{cases}
  ?^0 \mapsto & \Bool
  \end{cases}
  & \onslide<2->{= & \; s \circ id} \\
  s' & =
  \begin{cases}
  ?^0 \mapsto & \Bool \\
  ?^1 \mapsto & \Bool
  \end{cases}
  & \onslide<3->{= & \; s \circ \begin{cases}?^1 \mapsto \Bool\end{cases}} \\
  s'' & =
  \begin{cases}
  ?^0 \mapsto & \Bool \\
  ?^1 \mapsto & \Bool \\
  ?^2 \mapsto & ?^{24} \rightarrow \; ?^{25}
  \end{cases}
  & \onslide<4->{ = & \;
    s \circ
    \begin{cases}
      ?^1 \mapsto & \Bool \\
      ?^2 \mapsto & ?^{24} \rightarrow \; ?^{25}
    \end{cases}
  }
  \end{aligned}
  \]

\end{frame}

\begin{frame}
  % the other desirable property is idempotence. this is mainly just for convenience,
  % it's nicer if a solution does 'all its work in one go'.
  \frametitle

  \begin{center}Idempotence:\end{center}

  \[ s \circ s = s \]

\end{frame}

\begin{frame}[t]
  \frametitle

  \[
  \begin{aligned}
    & unify & : & \; Set \; Equation \rightarrow Substitution \\
    \onslide<2->{& unify(\{\}) & = & \; id \\}
    \onslide<3->{& unify(\{ B_1 = B_2, rest \}) & = & \; \\}
    \onslide<4->{& ~~ \texttt{if} \; B_1 = B_2 \; \texttt{then} \; unify(rest) \; \texttt{else} \; \bot \span \span \\}
    \onslide<5->{& unify(\{ C_1(x_1, \; \dots, \; x_n) = C_2(y_1, \; \dots, \; y_m), rest \}) & = & \; \\}
    \onslide<6->{& ~~ \texttt{if} \; C_1 = C_2 \; \texttt{then} \; unify(\{ \; x_i = y_i \; | \; i \in [1,\;n] \; \} \cup rest) \; \texttt{else} \; \bot \span \span \\}
    \onslide<7->{& unify(\{ ?^n = t, rest \}) & = & \; \\}
    \onslide<8->{& ~~ \texttt{let} \; s = \begin{cases}?^n \mapsto t \end{cases} \\}
    \onslide<9->{
      & ~~ \texttt{if} \; occurs(?^n, t) \; \texttt{then} \; \bot \; \texttt{else} \;
      s \circ unify(\{ \; s(t) = s(u) \; | \; (t = u) \in rest \; \}) \span \span \\
    }
    \onslide<10->{& unify(\{ s = t, rest \}) & = & \; unify(\{ t = s, rest \})}
  \end{aligned}
  \]

\end{frame}

\begin{frame}
  \frametitle

  Why the 'occurs' check?

\end{frame}

\begin{frame}
  \frametitle

  \onslide<+->{\[ ?^n = \Bool \]}
  \bigskip
  \onslide<+->{\[ s = \begin{cases}?^n \mapsto \Bool\end{cases} \]}
  \bigskip
  \[
  \bigskip
  \begin{aligned}
  \onslide<+->{& s(?^n) & = & \; s(\Bool) \\}
  \onslide<+->{& \Bool & = & \; \Bool}
  \end{aligned}
  \]

\end{frame}

\begin{frame}
  \frametitle

  \onslide<+->{\[ ?^n = \; ?^n \rightarrow \Bool \]}
  \bigskip
  \onslide<+->{\[ s = \begin{cases}?^n \mapsto \; ?^n \rightarrow \Bool \end{cases} \]}
  \bigskip
  \[
  \bigskip
  \begin{aligned}
  \onslide<+->{& s(?^n) & = & \; s(?^n \rightarrow \Bool) \\}
  \onslide<+->{& ?^n \rightarrow \Bool & = & \; (?^n \rightarrow \Bool) \rightarrow \Bool}
  \end{aligned}
  \]

\end{frame}

\begin{frame}
  \frametitle

  \onslide<+->{Performance considerations: \\}

  \begin{itemize}
  \item<+-> Substitution is slow
    \begin{itemize}
      \item<+-> Mutable variables for metas
      \item<+-> Union-Find data structure
    \end{itemize}
  \item<+-> Occurs check is slow
    \begin{itemize}
      \item<+-> Ignore it
      \item<+-> Defer it
    \end{itemize}
  \end{itemize}

\end{frame}

\section{Polymorphism}

% we're missing one last piece in our inference system

\begin{frame}
  \frametitle

  \[ \lam{x}{\lam{y}{x}} \]

\end{frame}

\begin{frame}
  % at the end of inference for this term we have unsolvable metas
  \frametitle

  \[
  \prftree[r]{\rn{(T-lam)}}
    { new(?^0) }
    {
      \prftree[r]{\rn{(T-lam)}}
        { new(?^1) }
        { \prftree[r]{\rn{(T-var)}}
            {}
            { \Gamma, x : \; ?^0, y : \; ?^1 \vdash x \Rightarrow \; ?^0 }
        }
        { \Gamma, x : \; ?^0 \vdash \lam{y}{x} \;\; \Rightarrow \;\; ?^1 \rightarrow \; ?^0 }
    }
    { \Gamma \vdash \lam{x}{\lam{y}{x}} \;\; \Rightarrow \;\; ?^0 \rightarrow \; ?^1 \rightarrow \; ?^0 }
  \]

\end{frame}

\begin{frame}
  \frametitle

  \[
  \begin{aligned}
  tyvar ::= & \\
  & \alpha \\
  & \beta \\
  & \dots \\ \\
  Type ::= & \\
  & \dots \\
  & tyvar
  \end{aligned}
  \]

\end{frame}

\begin{frame}
  \frametitle

  \[
  \begin{aligned}
  Scheme ::= & \\
  & Type \\
  & \forall tyvar . \; Scheme \\ \\
  Context ::= & \\
  & \bullet \\
  & Context , \; var : Scheme
  \end{aligned}
  \]

\end{frame}

\begin{frame}
  \frametitle

  \[ \forall \alpha\texttt{.} \; \alpha \rightarrow \alpha \]
  \[ \forall \alpha\texttt{.} \; \forall \beta\texttt{.} \; \alpha \rightarrow \beta \rightarrow \alpha \]

\end{frame}

\begin{frame}
  \frametitle

  \[
  \begin{aligned}
  & freetvs_{\texttt{Type}} & : & \; Type \rightarrow Set \; tyvar \\
  & freetvs_{\texttt{Type}} & = & \; \dots
  \end{aligned}
  \]

  \bigskip

  \[
  \begin{aligned}
  & freetvs_{\texttt{Scheme}} & : & \; Scheme \rightarrow Set \; tyvar \\
  & freetvs_{\texttt{Scheme}}(\forall \alpha \texttt{.} T) & = & \; freetvs_{\texttt{Scheme}}(T) - \{\alpha\} \\
  & freetvs_{\texttt{Scheme}}(T) & = & \; freetvs_{\texttt{Type}}(T)
  \end{aligned}
  \]

  \bigskip

  \[
  \begin{aligned}
  & freetvs_{\texttt{Context}}(\bullet) & = & \; \{\} \\
  & freetvs_{\texttt{Context}}(\Gamma , x : T) & = & \; freetvs_{\texttt{Scheme}}(T) \cup freetvs_{\texttt{Context}}(\Gamma)
  \end{aligned}
  \]
\end{frame}

\begin{frame}

  \[
  \begin{aligned}
  & instantiate & : & \; Scheme \rightarrow Type \\
  & instantiate(\forall \alpha \texttt{.} T) & = & \; new(?^n); \; instantiate(T)[\alpha := \; ?^n] \\
  & instantiate(T) & = & \; T
  \end{aligned}
  \]

\end{frame}

\begin{frame}
  \frametitle

  \[
  \begin{aligned}
  & solve : Type \rightarrow type \span \span \\
  & solve = \dots \span \span \\ \\
  & generalize & : & \; Context \rightarrow Type \rightarrow Scheme \\
  & generalize(\Gamma, T) & = & \\
  & ~~ \texttt{let} \; T' = solve(T) \span \span \\
  & ~~ \forall (freetvs_{\texttt{Type}}(T') - freetvs_{\texttt{Context}}(\Gamma)) \texttt{.} T' \span \span
  \end{aligned}
  \]

\end{frame}

\begin{frame}
  \frametitle

  \infrule[T-var]
    {}
    {\Gamma , \; x : T \vdash x \Rightarrow instantiate(T)}

\end{frame}

\newcommand{\letin}[3]{\texttt{let } #1 = #2 \texttt{ in } #3}

\begin{frame}
  \frametitle

  \[
  \begin{aligned}
  term ::= & \\
  & \dots \\
  & \letin{var}{term}{term}
  \end{aligned}
  \]

\end{frame}

\begin{frame}
  % 'let generalization' is a little controversial. simon pj et al thing that 'lets'
  % shouldn't have generalization, and only top level types should be generalized
  \frametitle

  \infrule[T-let]
    {\Gamma \vdash e \Rightarrow S  \andalso   \Gamma, x : generalize(\Gamma, S) \vdash b \Rightarrow T}
    {\Gamma \vdash \letin{x}{e}{b} \;\; \Rightarrow \;\; T}

\end{frame}

\newcommand{\letrecin}[3]{\texttt{letrec } #1 = #2 \texttt{ in } #3}

\begin{frame}
  \frametitle

  \[
  \begin{aligned}
  term ::= & \\
  & \dots \\
  & \letrecin{var}{term}{term}
  \end{aligned}
  \]

\end{frame}

\begin{frame}
  % 'let generalization' is a little controversial. simon pj et al thing that 'lets'
  % shouldn't have generalization, and only top level types should be generalized
  \frametitle

  \infrule[T-letrec]
    {
      new(?^n)   \andalso
      \Gamma, x : \; ?^n \vdash e \Rightarrow S   \andalso
      ?^n = S   \andalso
      \Gamma, x : generalize(\Gamma, S) \vdash b \Rightarrow T
    }
    {\Gamma \vdash \letrecin{x}{e}{b} \;\; \Rightarrow \;\; T}

\end{frame}

\begin{frame}
  \frametitle
  \onslide<+->{Performance improvements: \\}

  \begin{itemize}
  \item<+-> 'Batched' instantiation/generalization
  \item<+-> Generalization is slow
    \begin{itemize}
      \item<+-> Lambda / Let ranking
    \end{itemize}
  \end{itemize}
\end{frame}

\section{Further reading}

\begin{frame}
  \frametitle

  \begin{itemize}
  \item<2->
    Martelli, A., \& Montanari, U. (1982). An efficient unification algorithm. \textit{ACM Transactions on Programming Languages and Systems (TOPLAS)}, 4(2), 258-282.
  \item<3->
    Paterson, M. S., \& Wegman, M. N. (1978). Linear unification. \text{Journal of Computer and System Sciences}, 16(2), 158-167.
  \item<4->
    Milner, R. (1978). A theory of type polymorphism in programming. \textit{Journal of computer and system sciences}, 17(3), 348-375.
  \item<5->
    Damas, L., \& Milner, R. (1982, January). Principal Type-Schemes for Functional Programs. In \textit{POPL} (Vol. 82, pp. 207-212).
  \item<6->
    \url{http://dev.stephendiehl.com/fun/006_hindley_milner.html}
  \item<7->
    Kuan, G., \& MacQueen, D. (2007, October). Efficient type inference using ranked type variables. In \textit{Proceedings of the 2007 workshop on Workshop on ML} (pp. 3-14). ACM.
  \end{itemize}
\end{frame}

\end{document}
